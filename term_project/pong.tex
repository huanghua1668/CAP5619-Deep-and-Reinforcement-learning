\documentclass[12pt]{article}
\setlength{\parindent}{0pt} 
\setlength{\parskip}{10pt} % block paragraphs
\usepackage{algorithmicx}
\usepackage[T1]{fontenc}
\usepackage{algpseudocode}
\usepackage{graphicx}
\usepackage{amsmath}
\usepackage{bm}

\algdef{SE}[DOWHILE]{Do}{doWhile}{\algorithmicdo}[1]{\algorithmicwhile\ #1}%

\begin{document}

\title*{\centerline{\huge{Pong control problem using SARSA algorithm}}}
\author*{\centerline{Hua Huang}}%unnumbered centered head

\section{Why playing Pong is more difficult than we thought?}
When we think about playing Pong, we may consider it as something
easy if not trivial, we just need to bounce back the ball. The expert human performance
on Pong \cite{13atari} is -5 (while interestingly in Nature paper \cite{15atari}, 
it's giving as 9.3). Human performance is the average reward achieved from around 20
episodes of each game lasting a maximum of 5 min each, following around 2 h of practice
playing\cite{15atari}.\\
Based on the Gym package and the Pong\--V1 environment, and design the
features as cosine series with highest order of 3, so far the agent
can only achieve an average result of -13. Why does the agent fail 8
episodes out of 29 episodes? By inspecting frame by frame, we can
speculate that the stochastic dynamics of the environment may be one of the
biggest reasons.

A clip of the action and the following position difference of paddle of agent 
are listed in the figure 1. u stands for action up, and d stands for
action down. When the position difference is negative, means the
paddle moves up, and when the position difference is positive, means
the paddle moves down. In the figure, we can see that same action up
will have a large set of possible consequences, the paddle can move
\{-11, -10, -9, -4, -3, -1, 2\}, similarly, for action down, the paddle
can move \{-5, -2, 4, 6, 7, 10, 11, 12\}. For reference, the height of ball
is only 2.\\
\begin{figure}[h!]
\centering
\includegraphics[scale=0.8]{agent_paddle_action_dy.png}
\caption{action and its consequence}
\end{figure}

Why same action has so many different consequences? One of the reasons
is the complex system, the simulator is designed to be highly
unlinear. This is the reason why the deepmind keep record of four
consecutive frames to form the state, namely to take care of the
partial observability, we need to keep 4 consecutive frames to fully
observe the trajectories. 4 frames will impose a big challenge for
shallow learning, the dimension is very high. Another reason is the 
handling of the frames. In environement Pong\--V1, the same action is 
uniformly sampled from set \{2,3,4\}, namely same action is reapetedly 
carried out 2\--4 times. To make the learning a bit little easier,
environment is hereafter changed to PongDeterministic\--V0, in which action is
repeated 4 times. This environment is also adopted in deepmind's
paper, so for comparison reason, it's also valid to make this change.


\section{A summary of current try}
Since the bar is already high (the deep RL can acheieve a score as
high as 19), before we try feature selection, it's necessary to
enhance the performance of the current try, and after that, we can
adopt feature selection to accelerate the trainning. 

By inspecting the magnitude of the weights, we can speculate that the agent
paddle does not have a high\--order effect, so it might not degrade
the performance if we only maintain low\--order for agent paddle. To
test this idea, a new state is designed, $s=[dx_{ball}, dy_{ball}, x_{ball},
y_{ball}, y_{agent}]$, since we only have dimension 5, order of 10 is
tried here. The total feature amount is 320k. The increase of order
boost the performance significantly, the current average score is
-2.2. By inspecting the weights, high\--order features involving ball
actions have large\--magnitude weights. To further test this idea,
order of 7 is tested, and the best performance so far is only -9.62. 
Again, we can see the ball movement indeed has a high\--order effect.
\begin{figure}[h!]
\centering
\includegraphics[scale=0.8]{learning_curve_order_effect.png}
\caption{order effect on learning}
\end{figure}

\begin{thebibliography}{9}
    \bibitem{13atari}
        Volodymyr Mnih et al.,\textit{Playing Atari Using Deep
        Inforcement Learning},  arXiv:1312.5602.

    \bibitem{15atari}
                Mnih, V. et al. \textit{Human-level control through
                deep reinforcement learning}.Nature 518, 529-533 (2015).
\end{thebibliography}

\end{document}
